\documentclass[a4paper]{article}
\usepackage[PUTF,T1]{fontenc}
\usepackage[utf8]{inputenc}
\usepackage{tabularx}
\usepackage[table]{xcolor}
\usepackage{graphicx}
\usepackage{datetime}
%\usepackage{libertine}
\usepackage{ngerman}
\usepackage{fancyhdr}
%\usepackage{xpatch}
%\usepackage[forget]{qrcode}
%\usepackage{seqsplit}

\usepackage{avant} %use helvetica?


\usepackage{geometry}
 \geometry{
 a4paper,
 total={170mm,257mm},
 left=20mm,
 top=0mm,
 bottom=35mm,
  heightrounded,
 }
 
 %Thanks, Ulrike: https://github.com/u-fischer/putfenc
 \makeatletter
%qrcode wird gepatcht, damit es zur anderen Kodierung wechselt:
%\xpatchcmd\qrcode@i{\begingroup}{\begingroup\fontencoding{PUTF}\fontfamily{pdf}\selectfont}{}{\show\failed}

% Unterdrücken der aux-Ausgabe:
%\def\qr@writebinarymatrixtoauxfile#1{}%
\makeatother

\DeclareFontFamily{PUTF}{pdf}{}%
\DeclareFontShape{PUTF}{pdf}{m}{n}{ <-> ecrm1000 }{}%
\DeclareFontSubstitution{PUTF}{pdf}{m}{n}%


 \definecolor{papercolor}{rgb}{1,1,1}
 \definecolor{bgcolor}{rgb}{0.137, 0.122, 0.125}
 \definecolor{textcolor}{rgb}{0.9,0.9,0.9}
 
 \pagecolor{bgcolor}


\usepackage[xetex,%
            colorlinks=true,linkcolor=blue,citecolor=blue,%
            anchorcolor=red,urlcolor=blue,bookmarks=true,%
            bookmarksopen=true,bookmarksopenlevel=0,plainpages=false,%
            bookmarksnumbered=true,hyperindex=false,pdfstartview=,%
            pdfauthor={StuStaPay},%
            pdftitle={StuStaPay Bon},%
            %pdfsubject={StuStaPay Beleg}%
]{hyperref}


% for deterministic build
\special{pdf:trailerid [
    <00112233445566778899aabbccddeeff>
    <00112233445566778899aabbccddeeff>
]}

\begin{document}
    \color{textcolor}


    \pagestyle{fancy}
    \fancyhf{}
    \renewcommand{\headrulewidth}{0pt}
    \fancyhead{} % clear all header fields
    \fancyfoot[C]{\includegraphics[width=3cm]{logo_dark}}%\\\textcolor{textcolor}\thepage}
    
    \noindent
    \makebox[17cm]{
    \includegraphics[width=21cm,trim={15.5cm 0 10.15cm 0},clip]{stustaculum23_wristband.pdf}
    }
    \vspace{1em}
    
        
    \begin{center}
     \Huge\textsf{Rechnung}
    \end{center}
    \vspace{2em}
    
    \textbf{Rechnung Nr. \VAR["{:010}".format(order.id)]} \hspace{\fill}  \VAR[order.booked_at.date()|latex]

    % Addresse und Steuer ID
    \begin{flushright}
        \VAR[config.issuer|latex] \\
        \VAR[config.address.strip()|latex] \\
        USt-IdNr. \VAR[config.ust_id|latex] \\
    \end{flushright}

    % Line Items in the Rechnung
    \vspace{1cm}
    \newcolumntype{R}{>{\raggedleft\arraybackslash}X}
    \rowcolors{2}{}{bgcolor!80}
    \begin{tabularx}{\textwidth}{ p{1cm} p{4cm} R R X }
        ~                    & \textbf{Produkt}        & \textbf{Einzelpreis}   & \textbf{Gesamt}              &                       \\
        \hline
        \BLOCK[ for item in order.line_items ]
        \VAR[item.quantity|latex]x & \VAR[item.product.name|latex] & \VAR[item.product_price|money] & \VAR[item.total_price|money] & [\VAR[item.tax_name|latex]] \\
        \BLOCK[endfor]
    \end{tabularx}


    % Gesamtsumme
    \vspace{1cm}
    \hspace{\fill} \textbf{Summe EUR} \textbf{\VAR[order.total_price|money]€}

    %Steuerstatz
    \vspace{1cm}
    \rowcolors{0}{}{}
    \begin{tabularx}{\textwidth}{ RRRR }
        \textbf{MwSt.\%}                                         & \textbf{Brutto}                & \textbf{Netto}                   & \textbf{MwSt.}                 \\
        \hline
        \BLOCK[for tax_rate in tax_rate_aggregations ]
        \VAR[tax_rate.tax_name|latex]=\VAR[tax_rate.tax_rate|percent] & \VAR[tax_rate.total_price|money] & \VAR[tax_rate.total_no_tax|money] & \VAR[tax_rate.total_tax|money] \\
        \BLOCK[endfor]
        \hline
        ~                                                        & \VAR[order.total_price|money]    & \VAR[order.total_no_tax|money]    & \VAR[order.total_tax|money]    \\
    \end{tabularx}

    % sonstige Infos
    \vspace{1cm}
    \begin{center}
    \footnotesize
    \begin{tabular*}{10cm}{ p{5cm} p{6cm} }
        Zahlmethode:        & \VAR[order.payment_method.name|latex]    \\
% We use the TSE start and stop time instead of our order booking time
%         Vorgangszeitpunkt:  & \VAR[order.booked_at.astimezone()]  \\
        Kassierer:          & \VAR[order.cashier_id|latex]  \\
        Seriennummer Kasse: & \VAR[order.till_id|latex]     \\
        \BLOCK[if order.signature_status == "done"]
        Transaktionsnummer: & \VAR[order.tse_transaction|latex] \\
        Signaturzähler:     & \VAR[order.tse_signaturenr|latex] \\
	Signatur:           & \VAR[order.tse_signature|latex] \\
	%Signatur:           & \seqsplit{\VAR[order.tse_signature|latex]} \\
        Start-Zeit:         & \VAR[order.tse_start|latex] \\
        Stop-Zeit:          & \VAR[order.tse_end|latex] \\
        Public-Key          & \VAR[order.tse_public_key|latex] \\
        %Public-Key          & \seqsplit{\VAR[order.tse_public_key|latex]} \\
        Hash-Algorithmus:   & \VAR[order.tse_hashalgo|latex] \\
        Zeitformat:         & \VAR[order.tse_time_format|latex] \\
        QR-Code             & \VAR[order.tse_qr_code_text|latex] \\
        \BLOCK[else]
        TSE                 & ausgefallen \\
        \BLOCK[endif]
    \end{tabular*}

    \vspace{1em}
    \vspace{3cm}
    %\qrset{height=3cm}
    %\color{bgcolor}
    %\fcolorbox{textcolor}{textcolor}{\qrcode[nolinks]{\VAR[order.tse_qr_code_text|latex]}}
    \end{center}

\end{document}
