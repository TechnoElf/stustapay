\documentclass[a4paper,12pt]{fseibrief}
\usepackage{rechnung}
\usepackage[utf8]{inputenc}
\usepackage[ngerman]{babel}
\usepackage{eurosym}
\usepackage{seqsplit}
\usepackage{array}
\usepackage{multirow}
%\usepackage{draftwatermark} %Wasserzeichen

%\SetWatermarkText{\Huge{KEINE RECHNUNG}}
\begin{document}

\subject{\bf Beleg 24-GEX-000000}
\signature{}
\Datum{\today}

\begin{letter}{}

\opening{}

% [N] bei angabe von Nettopreisen, bei [B] wird der preis als brutto interpretiert
\begin{Rechnung}[B] % ohne * gibt's keine Artielnummer
\Euro
\PositionAus
\TrennerAus
\Steuersatz{19}{0} % der erste ist der normale Steuersatz, der zweite der ermäßigte. Wenn die Rechnung nur 7% hat sieht es schöner aus wenn man es umdreht
\EinzelpreisAus % dann wird der Preis als Gesamtpreis interpretiert und nicht mit der Anzahl multipliziert
	\Artikel[]{1}{Bier}{10}
\end{Rechnung}
\begin{tabular}{|>{\raggedleft}p{0.1\linewidth}|p{0.6\linewidth}|>{\raggedleft}p{0.2\linewidth}|}
	\hline
	\small{Anzahl} & \small{Beschreibung} & \small{Gesamtpreis} \tabularnewline
	\hline
	1 & Bier & 10,00€ \tabularnewline
	\hline
	\hline
	\multicolumn{2}{|l|}{Gesamtsumme} & \textbf{10,00€} \tabularnewline
	\hline
	\multicolumn{2}{|l|}{inkl. 19\% MwSt.} & 1,00€ \tabularnewline
	\hline
\end{tabular}

\end{letter}
\end{document}
